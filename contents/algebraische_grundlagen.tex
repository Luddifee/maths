\section{Grundlagen aus der Algebra}
\subsection{Mengen}
\begin{definition}
    Ein Objekt, das aus einer Zusammenfassung von einzelnen Elementen hervorgeht heißt Menge.
\end{definition}

\subsubsection{Kardinalzahlen}
\begin{definition}
    Eine natürliche Zahl, die die Anzahl der Elemente einer Menge angibt, heißt Kardinalzahl. Diese Gesamtanzahl von Elementen in einer Menge heißt Kardinalität oder Mächtigkeit.
\end{definition}

\subsubsection{Ordinalzahlen}
\begin{definition}
    Eine natürliche Zahl, die die Position eines Elements in einer Menge angibt, heißt Ordinalzahl.
\end{definition}

\subsubsection{Teilmenge}
\begin{definition}
    Eine Teilmenge $M$ ist eine Menge, die alle Elemente einer anderen Menge enthält. Enthält eine Menge $M_1$ alle Elemente einer anderen Menge $M_2$, so wird dies durch $M_1 \subset M_2$ oder $M_2 \supset M_1$ ausgedrückt.
\end{definition}

\subsubsection{Vereinigungsmenge}
\begin{definition}
    Die Menge $M = M_1 \cup M_2$ heißt Vereinigungsmenge. Dabei gilt für $M_1 = \{a;b\}$ und $M_2 = \{b;c\}$, dass $M = \{a;b;c\}$.
\end{definition}

\subsubsection{Schnittmenge}
\begin{definition}
    Die Menge $M = M_1 \cap M_2$ heißt Schnittmenge. Dabei gilt für $M_1 = \{a;b\}$ und $M_2 = \{b;c\}$, dass $M = \{b\}$.
\end{definition}

\subsubsection{Differenzmenge}
\begin{definition}
    Die Menge $M = M_1 \setminus M_2$ heißt Differenzmenge. Dabei gilt für $M_1 = \{a;b\}$ und $M_2 = \{b;c\}$, dass $M = \{a\}$.
\end{definition}

\subsubsection{Komplementärmenge}
\begin{definition}
    Die Menge $M = \overline{M_2} = M_1 \setminus M_2$ für $M_2 \subset M_1$ heißt Komplementärmenge. Dabei gilt für $M_1 = \{a;b;c\}$ und $M_2 = \{a;b\}$, dass $M = \{c\}$.
\end{definition}

\subsubsection{Symmetrische Differenz}
\begin{definition}
    Die Menge $M = M_1 \, \Delta \, M_2$ heißt symmetrische Differenz. Dabei gilt für $M_1 = \{a;b\}$ und $M_2 = \{b;c\}$, dass $M = \{a;c\}$.
\end{definition}

\subsubsection{Kartesisches Produkt}
\begin{definition}
    Die Menge $M = M_1 \times M_2$ heißt kartesisches Produkt. Dabei gilt für $M_1 = \{a;b\}$ und $M_2 = \{b;c\}$, dass $M = \{(a;b);(a;c);(b;b);(b;c)\}$.
\end{definition}

\subsection{Zahlenmengen}
\subsubsection{Menge der natürlichen Zahlen}
\begin{definition}
    Die Menge $\mathbb{N} = \{1;2;3;4;\dots\}$ oder $\mathbb{N} = \{0;1;2;3;\dots\}$ heißt Menge der natürlichen Zahlen.
\end{definition}

\subsubsection{Menge der ganzen Zahlen}
\begin{definition}
    Die Menge $\mathbb{Z} = \{\dots;-3;-2;-1;0;1;2;3;\dots\}$ heißt Menge der ganzen Zahlen. 
\end{definition}

\subsubsection{Menge der rationalen Zahlen}
\begin{definition}
    Die Menge $\mathbb{Q}$, die ausschließlich alle Zahlen enthält, die als Division zweier ganzer Zahlen $z_1$ und $z_2 \neq 0$ durch $\displaystyle\frac{z_1}{z_2}$ dargestellt werden können, heißt Menge der rationalen Zahlen.
\end{definition}

\subsubsection{Menge der reellen Zahlen}
\begin{definition}
    Die Menge $\mathbb{R}$, die ausschließlich alle auf einem eindimensionalen Zahlenstrahl dargestellbaren Zahlen enthält, heißt Menge der reellen Zahlen.
\end{definition}

\subsection{Logische Ausdrücke}
\begin{definition}
    Ein mathematischer Ausdruck, der einen Wahrheitswert zum Ergebnis hat, heißt logischer Ausdruck.
\end{definition}
Beispiele für logische Ausdrücke sind $(a = b)$, $(a \neq b)$, $(a \in B)$ oder $(A \subset B)$.

\subsection{Terme}
\begin{definition}
    Zahlen und Variablen sowie Verbindungen dieser heißen Term. Terme mit gleichen Variablen heißen gleichartig. Zur Gruppierung von Termgliedern werden Klammern verwendet.
\end{definition}

\subsection{Termumformungen}
\begin{definition}
    Eine Veränderung eines Terms, bei der das Ergebnis gleichbleibend ist, heißt Termumformung.
\end{definition}


